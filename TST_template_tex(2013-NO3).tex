\documentclass[10.5pt,compsoc]{TsT}
\usepackage{graphicx}
\usepackage{footmisc}
\usepackage{subfigure}
\usepackage{url}
\usepackage{multirow}
\usepackage[noadjust]{cite}
\usepackage{amsmath,amsthm}
\usepackage{amssymb,amsfonts}
\usepackage{booktabs}
\usepackage{color}
\usepackage{ccaption}
\usepackage{booktabs}
\usepackage{float}
\usepackage{fancyhdr}
\usepackage{caption}
\usepackage{xcolor,stfloats}
\usepackage{comment}
\setcounter{page}{1}
\graphicspath{{figures/}}
\usepackage{cuted}  %flushend,
\usepackage{captionhack}
\usepackage{epstopdf}
%\usepackage[lite,subscriptcorrection,slantedGreek,nofontinfo]{mtpro2}

%===============================%
\headevenname{\zihao{-5}{\textbf{\emph{Tsinghua Science and Technology, June}}} 2013, 18(3): 000-000}%
\headoddname{{\sf Wang Zhenyang et al.:}\quad {\textbf{\emph{Attention to Difficult Instances: A Cascade Coarse to Fine Network Architecture for Semantic Segmentation}}}}%
%footnote use of *
\renewcommand{\thefootnote}{\fnsymbol{footnote}}
\setcounter{footnote}{0}
\renewcommand\footnotelayout{\normalsize}

\newtheoremstyle{mystyle}{0pt}{0pt}{\normalfont}{1em}{\bf}{}{1em}{}
\theoremstyle{mystyle}
\renewcommand\figurename{Fig.~}
\renewcommand{\thesubfigure}{(\alph{subfigure})}
\newcommand{\upcite}[1]{\textsuperscript{\cite{#1}}}
\renewcommand{\labelenumi}{(\arabic{enumi})}
\newcommand{\tabincell}[2]{\begin{tabular}{@{}#1@{}}#2\end{tabular}}
\newcommand{\abc}{\color{white}\vrule width 2pt}


\newtheorem{assumption}{\textbf{Assumption}}
\newtheorem{definition}{\textbf{Definition}}
\newtheorem{lemma}{\textbf{Lemma}}
\newtheorem{theorem}{\textbf{Theorem}}
\newtheorem{proposition}{\textbf{Proposition}}
\newtheorem{corollary}{\textbf{Corollary}}






%\newcommand{\TODO}[1]{\textbf{TODO: #1}}

\addtolength{\abovecaptionskip}{-2mm}
\addtolength{\belowcaptionskip}{-2mm}

%%% aanpassing van cite aan journal style
\makeatletter
%\def\@cite#1#2{\textsuperscript{[{#1\if@tempswa, #2\fi}]}}
\renewcommand{\@biblabel}[1]{[#1]\hfill}
\makeatother

\begin{document}

\thispagestyle{empty}

\begin{strip}\zihao{3}
\noindent
\\ \textbf{Template for Preparation of Manuscripts for \\ \emph{Tsinghua Science and Technology}}
\vskip 6mm
\zihao{5}

\noindent
This template is to be used for preparing manuscripts for submission to \emph{Tsinghua Science and Technology}. Use of this template will save time in the review and production processes and will expedite publication. However, use of the template is not a requirement of submission. Do not modify the template in any way (delete spaces, modify font size/line height, etc.).
\vspace{180mm}
\end{strip}
\clearpage

\hyphenpenalty=50000

\makeatletter
\newcommand\mysmall{\@setfontsize\mysmall{7}{9.5}}

%%%%%%%%%%%%%%%%%%%%%%%
\newenvironment{tablehere}
  {\def\@captype{table}}
  {}
\newenvironment{figurehere}
  {\def\@captype{figure}}
  {}
%%%%%%%%%%%%%%%%%%%%%%%
%%%%%%%%%%%%%%%%%%%%%%%%%%%%%%%%

\thispagestyle{plain}%
\thispagestyle{empty}%

\let\temp\footnote
\renewcommand \footnote[1]{\temp{\zihao{-5}#1}}
{}
\vspace*{-40pt}

%{\hbox
\noindent{\zihao{5-}\textbf{\scalebox{0.89}[1.0]{\makebox[5.6cm][s]{%
TSINGHUA SCIENCE AND TECHNOLOGY}}}}

\vskip .2mm
{\zihao{5-}
\textbf{
\hspace{-5mm}
\scalebox{1}[1.0]{\makebox[5.6cm][s]{%
I\hspace{0.70pt}S\hspace{0.70pt}S\hspace{0.70pt}N\hspace{0.70pt}{\color{white}%
l\hspace{0.70pt}l\hspace{0.70pt}}1\hspace{0.70pt}0\hspace{0.70pt}0\hspace{0.70pt%
}7\hspace{0.70pt}-\hspace{0.70pt}0\hspace{0.70pt}2\hspace{0.70pt}1\hspace{0.70pt%
}4\hspace{0.70pt}{\color{white}l\hspace{0.70pt}l\hspace{0.70pt}}0\hspace{0.70pt}%
?\hspace{0.70pt}/\hspace{0.70pt}?\hspace{0.70pt}?\hspace{0.70pt}{\color{white}%
l\hspace{0.70pt}l\hspace{0.70pt}}p\hspace{0.70pt}p\hspace{0.70pt}?\hspace{0.70pt}?\hspace{0.70pt}?%
-\hspace{ 0.70pt}?\hspace{0.70pt}?\hspace{0.70pt}?}}}

\vskip .2mm\noindent
{\zihao{5-}\textbf{\scalebox{1}[1.0]{\makebox[5.6cm][s]{%
V\hspace{0.8pt}o\hspace{0.8pt}l\hspace{0.8pt}u\hspace{0.8pt}m\hspace{0.8pt}%
e\hspace{0.6em}1\hspace{0.8pt}8,\hspace{0.6em}N\hspace{0.8pt}u\hspace{0.8pt}%
m\hspace{0.8pt}b\hspace{0.8pt}e\hspace{0.8pt}r\hspace{0.6em}3,\hspace{0.6em}%
J\hspace{0.8pt}u\hspace{0.8pt}n\hspace{0.8pt}e%
\hspace{0.6em}2\hspace{0.8pt}0\hspace{0.8pt}1\hspace{0.8pt}3}}}}


\begin{strip}
{\center \vskip 3mm
{\zihao{3}\textbf{
Attention to Difficult Pixels: A Cascade Coarse to Fine Network Architecture for Semantic Segmentation
}}
\vskip 9mm}

{\center {\sf \zihao{5}
Wang Zhenyang, Deng Zhidong$^*$, and Wang Shiyao
}
\vskip 5mm}
%{\center \zihao{-5}{\textbf{
%1.~School of Computer Science, China University of Geosciences, Wuhan 430074, China;  \\
%2.~Shandong Provincial Key Laboratory of Computer Network, Jinan 250014, China; \\
%3.~School of Electronic Engineering Naval University of Engineering, Wuhan 430033, China\\
%}}}
%
%\vskip 5mm

\centering{
\begin{tabular}{p{160mm}}

{\zihao{-5}
\linespread{1.6667} %
\noindent
\bf{Abstract:} {\sf
Scene labeling, based on semantic segmentation, is a fundamental topic in computer vision. The goal is to assign each pixel in the image a category label. Convolutional neural networks, especially the fully convolutional neural networks, have attracted increasing attention on semantic segmentation due to the powerful capabilities of extracting hierarchical features. Since it is required to learn to make dense predictions for each pixel, a simple network is hardly to obtain considerable performances on different scenes. In this paper, we propose a novel semantic segmentation network called HMNet, which aims to pay more attention to the hard examples. The network has three branches, where the first branch produces coarse output predictions, and the second branch selects the hard examples which will be fed to the last branch. All above branches focus on their own objectives and collaboratively learn to predict from coarse to fine inference. Since the semantic segmentation dataset contains a large number of relatively easy samples and some hard ones, HMNet is encouraged to select these hard examples to make further predictions which is help to improve the final performance. In order to evaluate predicting performance of the proposed HMNet, we conduct experiments on two public datasets including Sift Flow and Stanford Background Dataset. We show that the three branches can be trained in an end-to-end manner and the experimental results show that compared to all existing models, our HMNet consistently yields the best performance, with accuracy of 91.6\% and 89.7\%, respectively.
}
\vskip 4mm
\noindent
{\bf Key words:} {\sf 
semantic segmentation; online hard example mining; convolutional neural network
}}

\end{tabular}
}
\vskip 6mm

\vskip -3mm
\zihao{6}\end{strip}


\thispagestyle{plain}%
\thispagestyle{empty}%
\makeatother
\pagestyle{tstheadings}

\begin{figure}[b]
\vskip -6mm
\begin{tabular}{p{44mm}}
\toprule\\
\end{tabular}
\vskip -4.5mm
\noindent
\setlength{\tabcolsep}{1pt}
\begin{tabular}{p{1.5mm}p{79.5mm}}
$\bullet$& Wang Zhenyang, Deng Zhidong, Wang Shiyao are with the Department of Computer Science, Tsinghua University, Beijing 100084, China. E-mail: crazycry2010@gmail.com, michael@tsinghua.edu.cn, sy-wang14@mails.tsinghua.edu.cn \\
$\sf{*}$&
To whom correspondence should be addressed. \\
          &          Manuscript received: 2017-09-20; revised: year-month-day; accepted: year-month-day

\end{tabular}
\end{figure}\zihao{5}



%\vspace{3.5mm}
\section{Introduction}
\label{s:introduction}
\noindent

Semantic segmentation, also known as scene labeling, is one of the fundamental research topics in computer vision. 
The goal of semantic segmentation is to identify and assign each pixel in the image with a category. 
This requires a complete understanding of the semantic information of the entire image. 
That means, for the testing image, it needs to predict the label of each object, and it is also very important to determine the boundary of each object in pixel level. 
Semantic segmentation has a strong application requirement in the field of environment perception and autonomous self driving car.



\section{Related Works}
\label{s:Related}
\noindent
semantic segmentation aims to relate one semantic class (road, water, sea etc.) to each pixel of the input image. Both the global and local features have great impacts on the final performance of this task. Consequently, in terms of feature representations, the mainstream approaches can be divided into traditional hand-craft features and deep features based on deep neural networks. Particularly, driven by powerful convolutional neural networks that yield hierarchies of features, this pixel-level prediction tasks have achieved great progress.

In recent years, traditional methods have obtained several solutions on image segmentation. Considering the context information, several methods rely on MRF, CRF or other types of graphical models to ensure consistency of labeling [12], [13], [14], [15], [16], [17]. Besides, most methods employ pre-segmentation in order to produce super-pixels or segmented candidates, and extract features from these individual segments along with the combinations of adjacent segments. The graphical model is able to predict the most consistent set of segments about the given image.

Meanwhile, the neural networks, especially convolutional neural networks, are widely applied to the semantic segmentation since they are becoming increasingly more attractive and powerful. Some previous work uses the convolution neural networks to model the pixels of images directly. [19] is the first work that use CNNs for this semantic segmentation. They propose a multi-scale convolution neural network, which extracts the feature representations from different scales of local regions. The experimental results show that the network has capacities of learning texture, shape and domain information implicitly, and achieves better performance than traditional hand-craft features. In addition, the network is also able to generalized to the RGB-D images [20]. To ensure a good visual coherence and a high class accuracy, [21] propose a method to capture long range (pixel) label dependencies in images. They use a recurrent architecture for convolutional neural networks to capture a long range label dependency while keeping a tight control over the capacity. The procedure is based on supervised deep learning strategies. [22-24] train the parametric CNNs by sampling image patches, which speeds up the training time dramatically. Howevers, they find that patch-based CNNs suffer from local ambiguity problems. [22] estimate the global potential in a non-parametric framework and propose a large margin based CNN metric for better global potential estimation. [23-24] introduce quaddirectional 2D Recurrent Neural Networks to model the long range dependencies among pixels which is able to embed the global image context into the compact local representation and significantly enhance their discriminative power.

At the same time, the researchers attempt to use the pre-trained convolution neural networks for semantic segmentation. [25] obtain the local and proximal features by using the ConvNet while distant and global features are produced from Alex-net[26]. These above features are further aggregated to predict the categories. Diff from these methods, [27] present a fully convolutional network which is able to take input of arbitrary size and produce correspondingly-sized output with efficient inference and learning. They use the CNNs trained on ImageNet as a feature extractor and transfer their learned representations by fine-tuning on the task-specific datasets. Additionally, the predicted score maps are upsampled to the input dimensions by deconvolution layers. [28] also used pre-trained deep convolution neural networks in order to produce pixel-level labels. They believe that DCNN is good at extracting hierarchical features rather than sufficiently localizing for accuracy objects. Hence, they propose deep networks by combining the responses at the final DCNN layer with a fully connected Conditional Random Field (CRF). The fully connected pairwise CRF has an ability to capture fine edge details which boosts the final performance. In [29], it presents that the main problem of the current FCN-based models is lack of the use of global context to help to make decision. They exploit the capability of global context information by different-region-based context aggregation through a pyramid pooling module together with the proposed pyramid scene parsing network (PSPNet).

\section{Method}
\label{s:Method}
\noindent


Inspired by online hard example mining(OHEM) algorithm, we propose a cascade coarse to fine network architecture CCFNet. 
Section~\ref{s:arch} illustrates how to turn a ImageNet pre-trained model to a cascade coarse to fine semantic segmentation network CCFNet.
Taking the characteristics of semantic segmentation task into consideration, section~\ref{s:hard} introduces a hard instance mining method to learn the attention about the difficulty of each instance.
Section~\ref{s:hard} detailes the training and testing process of CCFNet. 
%The idea behind is  simple yet effective. 
%The segmentation datasets contain a large number of easy examples and a small number of hard examples.
%Paying more attention on these hard examples can make the training 
%Pay much more attention on these hard examples 

\subsection{Cascade Coarse to Fine Network Architecture}
\label{s:arch}
\noindent

We choose ResNet-50 pre-trained on ImageNet as our baseline model.
ResNet originally is designed for image classification which win ILSRVC 2015 competation and surpass the human performance on ImageNet dataset.
ResNet-50 is one of the version provided in experiments, which is faster than VGG-16 and more accurate than VGG-19.
Compared with ResNet-101, ResNet-50 is cheaper in computation resources and memory consumption, but can achieve a comparable accuracy.
Figure~\ref{f:arch}a  visualizes the network architecture of ResNet-50, which is composed by five stages with different configurations of layers and a classification stage.
We treat the ResNet-50 as a common feature extraction part of CCFNet by discarding the classification stage.


The architecture of CCFNet is shown as Fig.~\ref{f:arch}b, which is composed by three cascade network branches: a coarse segmentation branch as a baseline result, an attention branch to predict the difficulty of labeling each pixel instance, and a refine segmentation branch to refine the final segmentation results. 
These three branches share a common feature extraction network.

\bf{The feature extraction network} is a fundamental convolutional neural network.
A modified version of ResNet-50 is adopted by this paper.
For semantic segmentation task,  the context is important to predict the correct label of each pixel instance.
But it is difficult to determine the boundary of each pixel's context, since different objects may have different sizes.
And the problem gets more complicated when considering the various perspective of each images.
A simple yet effective method to solve this problem is to integrate multi-scale features for label predicting.
The residual error model itself has the property of extracting and integrating multi-scale features, which can be seen from Fig.~\ref{}.
From the unravelled view by Veit et al.~\ref{}, a two-unit ResNet is equivalent to an ensemble of four sub-networks with different receptive fields , as illustrated in Fig. ~\ref{}.
So the whole ResNet-50 can be expanded as a linearly growing ensemble of sub-networks, which can extract and integrate multi-scale features.

Besides, there are two improvements adopted by ResNet-50 to make it more suitable for segmentation task.
First, we only keep the first three pooling layers to preserve the resolution.
So the final resolution of prediction is 1/8 of the original input image. 
Secondly, we replace the convolutional layer in the last two stages with the dilated convolutions.
It can help to enlarge the reception field of predicted feature maps.
The modified ResNet-50 is used as feature extraction network in CCFNet.

\textbf{The coarse segmentation branch} is a baseline model for image semantic segmentation.
This branch is shown as the red part in Fig.~\ref{f:arch} b, we adopt a fully convolutional network(FCN) with two convolutional layers to predict the semantic classes for their regions.
Since the resolution is 1/8 of the original input image, the feature maps are up-sampled by bilinear interpolation.
Finally, a pixel-wise softmax loss is adopted to predict the probabilities of each pixel.


\bf{The attention branch} is proposed to learn a soft-attention, which is a one-channel feature map with the same resolution as the input image. 
It is mainly used to indicate the segment difficulty of each pixel.
The idea behind is simple yet effective. 
The segmentation datasets contain a large number of easy pixel instances and a small number of difficulty pixel instances.
Paying more attention on these difficulty pixel instances can make the training process converges faster and efficiently.
The attention branch shares the same feature extraction network as the coarse segmentation branch, and has a similar network structure.
The major different is that the attention branch is a two-category semantic segmentation network which is only used to predict the segment difficulty.
Just the same as the coarse segmentation branch, softmax layer is adopted again to generate a soft-attention weighting coefficient.


\bf{The refine segmentation branch} refines the segmentation results as the final network output.
This branch is more complicated compared with the first two branches.
The coarse segmentation branch is hard to segment all the pixels correctly.
So the pixel instances can be divided into two groups by the coarse segmentation branch.
The pixels which can be segment correctly by the coarse segmentation branch is denote as easy pixel instances, while the others are difficult ones.
A fine segmentation network is introduced to reclassification the difficult pixel instances.
Inspired by the PSPNet~\ref{}, pyramid pooling is adopted by the fine segmentation network to extract multi-scale features.
And the refine segmentation branch is a weighted summation of the coarse segmentation branch and the fine segmentation network, with the weighting coefficient predicted from the attention branch.
So an end-to-end learning branch is proposed to learn the final segmentation result directly.

The three branches are cascaded one by one, and constitute an end-to-end learning network with multiple loss functions.

%The simplest method to improve the segmentation result is by applying a reclassification for the difficult pixel instances.
%But an unbalanced sample distribution go against the convergence of the reclassification network.
%So we propose an end-to-end learning branch, the refine segmentation branch, to learn the final segmentation result directly.


\subsection{Attention to Difficulty Pixels}
\label{s:attention}
\noindent


Hard example mining is one of the commonly used training techniques for machine learning.
The traditional implement is a continuous iterative process which could be divided into two steps.
Firstly, the training model is fixed to screen out the difficult examples, and the training set is updated by adding a certain rate of difficult examples.
Secondly, in the fixed training set, the training model is re-trained.

In this paper, the two-step process of hard example mining is optimised to an end-to-end learning network framework.
For semantic segmentation task, each pixel should be assigned a category label.
So a single image contain enough training samples for hard example mining.
The attention branch is used to predict each pixel is easy or difficult for the coarse segmentation branch.
It makes the end-to-end learning possible by heuristically filtering out the hard examples online.
The final segmentation results relay more on the coarse segmentation branch if the pixel is predicted as an easy one.
Otherwise, the fine segmentation network takes up a larger proportion.
Inspired by online hard example mining, the attention branch with a heuristic strategy is introduced to CCFNet to predict the difficult of each pixel.
And the final segmentation results are promoted by hard sample selection.

During the training process, the attention branch is supervised by a label map with 0/1 values, indicating  easy or difficult for the pixel in corresponding position.
The attention branch is cascaded behind the coarse segmentation branch, so the 0/1 label map can be generated by a comparison between the prediction of the coarse segmentation branch and the segmentation ground truth.
0 indicates the pixel is misclassified by the coarse segmentation branch, while 1 represents a correctly prediction.
The 0/1 label map is used as the ground truth of the attention branch, supervising the attention branch to learn the difficulty of each pixel.

\subsection{Loss Function}
\noindent

CCFNet includes three branches, the coarse segmentation loss, the attention branch, and the refine segmentation branch. 
So there are three loss function respectively corresponding to those three branches.

The coarse segmentation branch and the refine one use two standard pixel-wise softmax losses for semantic segmentation.
The attention branch is also a p

There are three loss function in total, 
?????????????????3??????????????????????????????????????????????????????????????softmax????????????????????????????????????????????????????????????????????????????????????????????????????????????????????????????????????
????????????????????????????????????????????????????????????????????????????softmax????????????????????


\section{Experimental results}
\label{s:results}
\noindent
In this section, we will present the detailed information about the implementation of our HMNet. Moreover, we compare our model to the current state-of-the-art methods and achieves superior performances among the existing methods.

\subsection{Network Configuration}
\noindent
The implementation of our HMNet is based on public platform Caffe[30]. The training procedure uses stochastic gradient descent (SGD) algorithm via end-to-end learning. Our model adopt pre-trained models by [29] like most related work on semantic segmentation. The learning rate is initially set to 1e-4 and repeatedly decreased 2 times, corresponding to 3K iterations, 4K iterations, respectively. The momentum and weight decay are set to 0.9 and 0.0001.

Data argumentation is widely applied to semantic segmentation in order to avoid the over-fit. Various kinds of transformations are used to expand the training sizes so as to improve the generalization of the proposed networks. In our experiments, we also employ this kind of methods by expanding the input images 1-2 times and cropping 233*233 randomly for training.  

Larger input size and mini-batch can help improve the segmentation performance. Due to the limitation od both computation and memory, we only use 233*233 images and each mini-batch is 4.

\subsection{Test}
\noindent
Pixel-level semantic segmentation is usually measured using two accuracy measures: Pixel Accuracy and Class Accuracy. The average pixel accuracy is the percentage of the total number of pixels correctly classified in the test data set, which is usually measured by the intersection-over-union (IoU). The average category accuracy is the average of the correct rate for each category of pixel classification.

\subsection{Data Sets}
\noindent
We prove the effectiveness of our HMNet on two semantic segmentation datasets, which are SIFT Flow [11] and Stranford Background [12], respectively. The SIFT Flow dataset contains 2688 samples each has 256x256 pixels with RGB channels. 2488 images are used as a training set while the rest 200 images are used for testing. The dataset defines a total of 33 semantic categories, but the distribution of category is non-uniform.

The Stanford Background dataset contains 715 images with different image sizes, but no more than 320x240 pixels. According to the previous research and testing methods, this paper divides the dataset by 5x cross validation method, and 572 images are used as training samples and 143 samples as the test samples. The Stanford Background dataset contains eight semantic categories, and the category distribution is more balanced than the SIFT Flow dataset

First, we conduct several experiments on SIFT Flow dataset. The results are show in Figure 3 which consist of four columns. The first column presents the input images and second shows the semantic labels. In particular, the third column indicates hard pixels predicted by HMNet while the last column is the segmentation results of the network. As can be seen from the third and fourth columns of Figure 3, the HMNet network can indeed predict the hard pixel samples which are indicated in the yellow square box of Figure 3.

In addition, on the SIFT Flow dataset, we compare the HMNet to other segmentation methods, the experimental results which is shown in Table 1 prove that HMNet achieves an accuracy of 91.6\%, and outperform the current state-of-the-art results.

\subsection{Comparison Results}
\noindent
In order to verify the generalization of the online hard learning of the semantic segmentation, we test the HMNet on another dataset called Stanford Background by using the same architecture and configurations. Table 2 shows that on the Stanford Background dataset, HMNet achieved 89.7\% pixel average accuracy and 75.4\% class average correct rate. Some of the prediction results are shown in Fig. 4, which can be clearly seen Predict the edge information of the results.

\subsection{Results Visualization}
\noindent
This paper presents a semantic segmentation network structure called HMNet. In the semantic prediction of the image, the pixel samples with difficult to identify the images are predicted online. Finally, the secondary identification of these difficult pixels is classified, which improves the correct rate of the whole network segmentation. On the two data sets of SIFT Flow and Stanford Background, HMNet achieved the best segmentation results for two segmented data sets with 91.6% and 89.7% test accuracy rates, respectively.


\section{Conclusions}
\noindent
For example: The parallelization of cutoff pair interactions is mature on CPUs, and typically employs a voxel-based method.


\vskip 6mm
\noindent
References to the literature are cited by \emph{number in square brackets} at appropriate locations (\emph{before} a period, comma, etc.) in the text.

\vskip 6mm
\noindent
Examples:

\setlength{\hangindent}{18pt}
\noindent
  $\bullet$  	Negotiation research spans many disciplines [3].

\setlength{\hangindent}{18pt}
\noindent
  $\bullet$  	This result was later contradicted by Becker and Seligman [5, 6], who .......

\setlength{\hangindent}{18pt}
\noindent
  $\bullet$  	This effect has been widely studied [1-3, 7].

\setlength{\hangindent}{18pt}
\noindent
  $\bullet$  	......achieved until rather recently [11, 21, 22], with......

\setlength{\hangindent}{18pt}
\noindent
  $\bullet$  	......stage of cap formation (see Fig. 5 in Ref. [14]).

\vskip 2mm
\zihao{5}
\noindent
\textbf{Acknowledgements}
\vskip 2mm

\zihao{5--}
\noindent
This work was supported in part by the National Science Foundation of China (NSFC) under Grant Nos. 91420106, 90820305, and 60775040, and by the National High-Tech R\&D Program of China under Grant No. 2012AA041402.

\vskip 2mm
\zihao{5}
\noindent
\textbf{\zihao{5}References}
\vskip 2mm

\zihao{5-}
\noindent
The font is Times New Roman $\backslash$zihao\{5\--\}. This part is placed at the end of the manuscript. References should be numbered sequentially as they appear throughout the text. Only one publication should be given for each number. The list of references should only include papers that have been published or accepted by a named publication or recognized preprint server. Authors should ensure the accuracy and completeness of all references before submission. Please ensure references are given in the correct style as shown below in order to avoid delays in typesetting your article
\renewcommand\refname{\zihao{5}\textbf{References}}


\begin{thebibliography}{99}
\zihao{5-} \addtolength{\itemsep}{-1em}
\vspace {1.5mm}

\bibitem[1]{1}
Krizhevsky A, Sutskever I, Hinton G E. Imagenet classification with deep convolutional neural networks[C]//Advances in neural information processing systems. 2012: 1097-1105.

\bibitem[2]{2}
Simonyan K, Zisserman A. Very deep convolutional networks for large-scale image recognition[J]. arXiv preprint arXiv:1409.1556, 2014.

\bibitem[3]{3}
Szegedy C, Liu W, Jia Y, et al. Going deeper with convolutions[C]//Proceedings of the IEEE Conference on Computer Vision and Pattern Recognition. 2015: 1-9.

\bibitem[4]{4}
Zeiler M D, Ranzato M, Monga R, et al. On rectified linear units for speech processing[C]//Acoustics, Speech and Signal Processing (ICASSP), 2013 IEEE International Conference on. IEEE, 2013: 3517-3521.

\bibitem[5]{5}
Graves A, Mohamed A, Hinton G. Speech recognition with deep recurrent neural networks[C]//Acoustics, speech and signal processing (icassp), 2013 ieee international conference on. IEEE, 2013: 6645-6649.

\bibitem[6]{6}
Sermanet P, Eigen D, Zhang X, et al. Overfeat: Integrated recognition, localization and detection using convolutional networks[J]. arXiv preprint arXiv:1312.6229, 2013.

\bibitem[7]{7}
Sermanet P, Eigen D, Zhang X, et al. Overfeat: Integrated recognition, localization and detection using convolutional networks[J]. arXiv preprint arXiv:1312.6229, 2013.

\bibitem[8]{8}
He K, Zhang X, Ren S, et al. Spatial pyramid pooling in deep convolutional networks for visual recognition[C]//European Conference on Computer Vision. Springer International Publishing, 2014: 346-361.

\bibitem[9]{9}
Ioffe S, Szegedy C. Batch normalization: Accelerating deep network training by reducing internal covariate shift[J]. arXiv preprint arXiv:1502.03167, 2015.

\bibitem[10]{10}
He K, Zhang X, Ren S, et al. Deep residual learning for image recognition[C]//Proceedings of the IEEE Conference on Computer Vision and Pattern Recognition. 2016: 770-778.

\bibitem[11]{11}
Liu C, Yuen J, Torralba A. Sift flow: Dense correspondence across scenes and its applications[J]. IEEE transactions on pattern analysis and machine intelligence, 2011, 33(5): 978-994.

\bibitem[12]{12}
Gould S, Fulton R, Koller D. Decomposing a scene into geometric and semantically consistent regions[C]//Computer Vision, 2009 IEEE 12th International Conference on. IEEE, 2009: 1-8.

\bibitem[13]{13}
He X, Zemel R S. Learning hybrid models for image annotation with partially labeled data[C]//Advances in Neural Information Processing Systems. 2009: 625-632.

\bibitem[14]{14}
Russell C, Kohli P, Torr P H S. Associative hierarchical crfs for object class image segmentation[C]//Computer Vision, 2009 IEEE 12th International Conference on. IEEE, 2009: 739-746.

\bibitem[15]{15}
Kumar M P, Koller D. Efficiently selecting regions for scene understanding[C]//Computer Vision and Pattern Recognition (CVPR), 2010 IEEE Conference on. IEEE, 2010: 3217-3224.

\bibitem[16]{16}
Mathieu M, Henaff M, LeCun Y. Fast training of convolutional networks through ffts[J]. arXiv preprint arXiv:1312.5851, 2013.

\bibitem[17]{17}
Springenberg J T, Riedmiller M. Improving deep neural networks with probabilistic maxout units[J]. arXiv preprint arXiv:1312.6116, 2013.

\bibitem[18]{18}
Lempitsky V, Vedaldi A, Zisserman A. Pylon model for semantic segmentation[C]//Advances in neural information processing systems. 2011: 1485-1493.

\bibitem[19]{19}
Farabet C, Couprie C, Najman L, et al. Learning hierarchical features for scene labeling[J]. IEEE transactions on pattern analysis and machine intelligence, 2013, 35(8): 1915-1929.

\bibitem[20]{20}
Couprie C, Farabet C, Najman L, et al. Indoor semantic segmentation using depth information[J]. arXiv preprint arXiv:1301.3572, 2013.

\bibitem[21]{21}
Pinheiro P H O, Collobert R. Recurrent Convolutional Neural Networks for Scene Labeling[C]//ICML. 2014: 82-90.

\bibitem[22]{22}
Shuai B, Wang G, Zuo Z, et al. Integrating parametric and non-parametric models for scene labeling[C]//Proceedings of the IEEE Conference on Computer Vision and Pattern Recognition. 2015: 4249-4258.

\bibitem[23]{23}
Shuai B, Zuo Z, Wang G. Quaddirectional 2d-recurrent neural networks for image labeling[J]. IEEE Signal Processing Letters, 2015, 22(11): 1990-1994.

\bibitem[24]{24}
Shuai B, Zuo Z, Wang B, et al. Dag-recurrent neural networks for scene labeling[C]//Proceedings of the IEEE Conference on Computer Vision and Pattern Recognition. 2016: 3620-3629.

\bibitem[25]{25}
Mostajabi M, Yadollahpour P, Shakhnarovich G. Feedforward semantic segmentation with zoom-out features[C]//Proceedings of the IEEE Conference on Computer Vision and Pattern Recognition. 2015: 3376-3385.

\bibitem[26]{26}
Krizhevsky A, Sutskever I, Hinton G E. Imagenet classification with deep convolutional neural networks[C]//Advances in neural information processing systems. 2012: 1097-1105.

\bibitem[27]{27}
Long J, Shelhamer E, Darrell T. Fully convolutional networks for semantic segmentation[C]//Proceedings of the IEEE Conference on Computer Vision and Pattern Recognition. 2015: 3431-3440.

\bibitem[28]{28}
Chen L C, Papandreou G, Kokkinos I, et al. Semantic image segmentation with deep convolutional nets and fully connected crfs[J]. arXiv preprint arXiv:1412.7062, 2014.

\bibitem[29]{29}
Zhao H, Shi J, Qi X, et al. Pyramid Scene Parsing Network[J]. arXiv preprint arXiv:1612.01105, 2016.

\bibitem[30]{30}
Jia Y, Shelhamer E, Donahue J, et al. Caffe: Convolutional architecture for fast feature embedding[C]//Proceedings of the 22nd ACM international conference on Multimedia. ACM, 2014: 675-678.

\bibitem[31]{31}
Tighe J, Lazebnik S. Superparsing: scalable nonparametric image parsing with superpixels[C]//European conference on computer vision. Springer Berlin Heidelberg, 2010: 352-365.

\bibitem[32]{32}
Tighe J, Lazebnik S. Finding things: Image parsing with regions and per-exemplar detectors[C]//Proceedings of the IEEE conference on computer vision and pattern recognition. 2013: 3001-3008.

\bibitem[33]{33}
Socher R, Lin C C, Manning C, et al. Parsing natural scenes and natural language with recursive neural networks[C]//Proceedings of the 28th international conference on machine learning (ICML-11). 2011: 129-136.

\bibitem[34]{34}
Eigen D, Fergus R. Nonparametric image parsing using adaptive neighbor sets[C]//Computer vision and pattern recognition (CVPR), 2012 IEEE Conference on. IEEE, 2012: 2799-2806.

\bibitem[35]{35}
Singh G, Kosecka J. Nonparametric scene parsing with adaptive feature relevance and semantic context[C]//Proceedings of the IEEE Conference on Computer Vision and Pattern Recognition. 2013: 3151-3157.

\bibitem[36]{36}
Liang M, Hu X. Recurrent convolutional neural network for object recognition[C]//Proceedings of the IEEE Conference on Computer Vision and Pattern Recognition. 2015: 3367-3375.

\begin{strip}
\end{strip}

\begin{biography}[yourphotofilename.eps]
\noindent
\textbf{First A. Author}\ \  Photo. Biographies should be limited to one paragraph consisting of the following: sequentially ordered list of degrees, including years achieved; sequentially ordered places of employ concluding with current employment; associa-tion with any official journals or conferences; major profes-sional and/or academic achievements, i.e., best paper awards, research grants, etc.; any publication information (number of papers and titles of books published); current research interests; association with any professional associations.
\end{biography}

\begin{biography}[yourphotofilename.eps]
\noindent
\textbf{Second B. Author} Photo. Biographies should be limited to one paragraph consisting of the following: sequentially ordered list of degrees, including years achieved; sequentially ordered places of employ concluding with current employment; associa-tion with any official journals or conferences; major profes-sional and/or academic achievements, i.e., best paper awards, research grants, etc.; any publication information (number of papers and titles of books published); current research interests; association with any professional associations.
\end{biography}
\vskip 22mm
\begin{biography}[yourphotofilename.eps]
\noindent
\textbf{Third C. Author}  Photo. Biographies should be limited to one paragraph consisting of the following: sequentially ordered list of degrees, including years achieved; sequentially ordered places of employ concluding with current employment; associa-tion with any official journals or conferences; major profes-sional and/or academic achievements, i.e., best paper awards, research grants, etc.; any publication information (number of papers and titles of books published); current research interests; association with any professional associations.
\end{biography}

\begin{strip}
\end{strip}

\mbox{}
\clearpage
\clearpage
\zihao{5}
\noindent
\textbf{FIGURES. }Below each figure, there should be a figure legend. The font is Times New Roman $\backslash$zihao\{5-\}. A figure legend normally begins with a brief title describing its whole contents, and continues with a short description of each panel. The format of figure legends is ``Fig. 1\quad Figure legend". The preferred figure formats is TIFF or JPEG with a preferred resolution of \textbf{600dpi} relative to the final figure size. When figures are divided into parts, each part should be labeled with a lower-case (a), (b), and so on, in the same font size as used elsewhere in the figure. All lettering in figures should be in lower-case, except for the first letter of each label which should be capitalized. Use \textbf{a single space} to separate a number and its units. Ensure that the labels are sufficiently large and clear to be readable when the figure is reproduced in the print version of the journal. Figures are referred to in the manuscript as ``Fig. 1" and ``Fig. 1a", or ``Figure 1" at the beginning of a sen-tence. After a manuscript is accepted, we may ask the authors to provide high resolution figures. Failure to supply the file can significantly delay the publication of your work. If possible, please provide a set of high resolution figures along with your initial manuscript. The width of figures should better be 8, 12, or 17 cm, and the height should be less than \mbox{19 cm.}\\

\noindent
\textbf{TABLES.} The font is Times New Roman $\backslash$zihao\{5-\}. Above each table, there should be a brief title describing its contents.~Title format should follow~``\textbf{Table 1\quad Table title}".~All details, such as nonstandard abbreviations, and a description of standards of error analysis, should be included in footnotes. Tables are referred to in the manuscript as ``Table 1". The concise type tables are recommended.\\

\noindent
\textbf{EQUATIONS.} The font is Times New Roman $\backslash$zihao\{5\}. Equations and mathematical expressions are identified by parenthetical numbers, such as (1), and are referred to in the manuscript as ``Eq. (1)", or ``Equation (1)" at the beginning of a sentence.


  \end{document}


